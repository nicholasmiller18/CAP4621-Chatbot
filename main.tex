\documentclass[11pt,twocolumn]{article}

\usepackage{graphicx} % for pdf, bitmapped graphics files
\usepackage{times} % assumes new font selection scheme installed
\usepackage{titling}
\usepackage[a4paper, total={7in, 10in}]{geometry}
\usepackage{cuted}
\usepackage{capt-of}
\usepackage[format=plain,
            labelfont=it,
            textfont=it]{caption}
\usepackage{sectsty}
\subsectionfont{\normalfont\itshape}
\usepackage[compact]{titlesec}
\titleformat*{\section}{\Large\bfseries\sffamily}

\setlength\parindent{0pt}
\setlength{\parskip}{1em}

\title{\textbf{Creating A Chatbot Using A Bidirectional Neural Network}}
\author{Jonathan Weese\thanks{jonweese@ufl.edu} \and Dylan Attlesey\thanks{Dylanattlesey@ufl.edu} \and Nicholas Miller\thanks{namiller@ufl.edu}}

\providecommand{\keywords}[1]
{
  \textbf{Keywords:} #1
}
\providecommand{\concepts}[1]
{
  \textbf{Concepts:} #1
}

\begin{document}
\maketitle

\begin{abstract}
asdf
\end{abstract}

\keywords{asdf}

\concepts{$\bullet$\textbf{Benchmark}$\rightarrow$\textbf{Test system performance}}


\section{Introduction}
As technology advances, the dream of creating human-like AI arises.  Among the many complexities and problems needed to be solver to create a human-lie AI, this explores one tiny section under that umbrella, specifically dealing with responding to humans using text.  Enabled with trained neural networks, computers have the ability to create human-like responses.

Although there are many chatbots throughout the world, there are fewer that can create responses dynamically.  Most chatbots that are created check the input sentence through if statements and other methods.  This way of designing a chatbot means that the chatbot can only create responses that the engineer has coded into it.  This is allows for great control over the responses but restricts the chatbot to what the engineer has thought of.  The chatbot designed in this report can create a logical response to any input and the response is unpredictable, in contrast to the preprogrammed chatbot.  This allows for the computer to create the illusion that you can have a conversation about anything with it.
\section{Background/Related Work}
asdf
\subsection*{Hadoop MapReduce}
asdf
\subsection*{Spark MapReduce}
asdf
\section{Experiment}
asdf
\subsection*{Environment Setup}
asdf
\subsection*{Experiment Design}
In this project, our goal was to create a Reddit reply generating chatbot by using a neural network.  We have a large database of Reddit post along with their top responses.  We would have had our neural network generate responses to Reddit post, and then evaluate the quality of the responses by comparing their similarity with the actual top responses.

We would have made the comparison by using word embeddings to compare the similarity of words generated by the network to the words in the original reply.
\section{Data Collection}
The data used to train the chatbot designed in this report was taken from Reddit.  A 30 GB file filled with a month (January, 2015) of posts, comments, users, post ID's, and more was used.  This file was parsed and the posts were paired to the most up voted comment for that post.  This pair was placed into a SQL database.  After parsing the Reddit file and filling the SQL database, a total of roughly 4,700,000 pairs were created.
\section{Data Analysis}
asdf
\section{Discussion}
\subsection*{Reflection}
\subsection*{Future Work}
asdf 
\subsection*{Conclusion}
asdf
%%%%%%%%%%%%%%%%%%%%%%%%%%%%%%%%%%%%%%%%%%%%%%%%%%%%%%%%%%%%%%%%%%%%%%%%%%%%%%%%
\bibliographystyle{acm}
\bibliography{sample}




\end{document}
